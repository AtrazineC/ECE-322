\documentclass[12pt, letterpaper, titlepage]{article}
\usepackage[utf8]{inputenc}
\usepackage{geometry}
\usepackage{color,graphicx,overpic} 
\usepackage{fancyhdr}
\usepackage{amsmath,amsthm,amsfonts,amssymb}
\usepackage{mathtools}
\usepackage{hyperref}
\usepackage{multicol}
\usepackage{array}
\usepackage{float}
\usepackage{blindtext}
\usepackage{longtable}
\usepackage{scrextend}
\usepackage[font=small,labelfont=bf]{caption}
\usepackage[framemethod=tikz]{mdframed}
\usepackage{calc}
\usepackage{titlesec}
\usepackage{listings}
\usepackage[normalem]{ulem}
\usepackage{tabularx}
\usepackage{mathrsfs}
\usepackage{bookmark}
\usepackage{apple_emoji}
\usepackage{setspace}
\usepackage{ragged2e}
\usepackage{ltablex}
\usepackage{textcomp}

% \mathtoolsset{showonlyrefs}  
\allowdisplaybreaks

\newcolumntype{b}{X}
\newcolumntype{s}{>{\hsize=.25\hsize}X}
\newcolumntype{a}{>{\hsize=.5\hsize}X}

\definecolor{mycolor}{rgb}{0, 0, 0}

\geometry{top=2.54cm, left=2.54cm, right=2.54cm, bottom=2.54cm}
\setlength{\headheight}{20pt}
\setlength{\parskip}{0.3cm}
\setlength{\parindent}{1cm}

\newcommand{\B}{\includegraphics[height=1.5em, valign=B, raise=-0.2em]{BigB.png}} 

\pagestyle{fancy}
\fancyhf{}
\rhead{\B enjamin Kong | 1573684}
\lhead{\textit{ECE 322 Assignment 4}}
\rfoot{Page \thepage}

\begin{document} 
\onehalfspacing

\section*{Q1}
We know
\begin{equation}
    p_j = \sum_{i=1}^{n}p_{ij}p_i, \quad j = 1,2,\ldots,n.
\end{equation}
From the figure, we have
\begin{align}
    p_\text{A} &= 0.8p_\text{H}, \\
    p_\text{B} &= 0.7p_\text{A} + 0.5p_\text{E}, \\
    p_\text{C} &= 1.0p_\text{B} + 0.7p_\text{D}, \\
    p_\text{D} &= 0.5p_\text{E}, \\
    p_\text{E} &= 0.3p_\text{A}, \\
    p_\text{F} &= 0.6p_\text{C}, \\
    p_\text{G} &= 0.4p_\text{C} + 1.0p_\text{F} + 0.2p_\text{H} \text{, and} \\
    p_\text{H} &= 0.3p_\text{D}+ 1.0p_\text{G}.
\end{align}
We also have
\begin{equation}
    p_\text{A} +
    p_\text{B} +
    p_\text{C} +
    p_\text{D} +
    p_\text{E} +
    p_\text{F} +
    p_\text{G} +
    p_\text{H} = 1.
\end{equation}
With the aforementioned equations, we have a systems of equation that we can solve. Using software to solve these systems of equations, we get 
\begin{align}
    p_\text{A} &\approx 0.1592, \\
    p_\text{B} &\approx 0.1353, \\
    p_\text{C} &\approx 0.1520, \\
    p_\text{D} &\approx 0.0239, \\
    p_\text{E} &\approx 0.0477, \\
    p_\text{F} &\approx 0.0912, \\
    p_\text{G} &\approx 0.1918 \text{, and} \\
    p_\text{H} &\approx 0.0199.
\end{align}
Therefore we conclude the priority for testing should be
\begin{equation}
    \text{H} > \text{G} > \text{A} > \text{C} > \text{B} > \text{F} > \text{E} > \text{D}.
\end{equation}

\section*{Q2}
\subsection*{i)}
When the functional dependencies among the input and output variables is not known, we have 
\begin{align}
    \text{\#Test cases} &= \#A \times \#B \times \#C \times \#D \times \#E \times \#F \\
    &= 5 \times 5  \times 3 \times 5 \times 2 \times 3 \\
    &= 2250.
\end{align}

\subsection*{ii)}
When the functional dependencies among the input and output variables is known, we have 
\begin{align}
    \text{\#Test cases} &= \#X + \#Y + \#Z \\
    &= (\#A \times \#D \times \#E) + (\#B) + (\#C \times \#F) \\
    &= (5 \times 5 \times 2) + (5) + (3 \times 3) \\
    &= 64.
\end{align}
Example test cases are shown below. ``N/A'' represents an entry that can be anything.

\begin{centering}
\begin{tabularx}{\textwidth}{|X|X|X|X|X|X|X|}
    \caption{Test cases for output X} \\ \hline
    \textbf{Test case} & \textbf{A} & \textbf{B} & \textbf{C} & \textbf{D} & \textbf{E} & \textbf{F} \\ \hline
    1 & 0 & N/A & N/A & 7 & Y & N/A \\ \hline
    2 & 0 & N/A & N/A & 7 & N & N/A \\ \hline
    3 & 0 & N/A & N/A & 8 & Y & N/A \\ \hline
    4 & 0 & N/A & N/A & 8 & N & N/A \\ \hline
    5 & 0 & N/A & N/A & 9 & Y & N/A \\ \hline
    6 & 0 & N/A & N/A & 9 & N & N/A \\ \hline
    7 & 0 & N/A & N/A & 10 & Y & N/A \\ \hline
    8 & 0 & N/A & N/A & 10 & N & N/A \\ \hline
    9 & 0 & N/A & N/A & 11 & Y & N/A \\ \hline
    10 & 0 & N/A & N/A & 11 & N & N/A \\ \hline
    11 & 1 & N/A & N/A & 7 & Y & N/A \\ \hline
    12 & 1 & N/A & N/A & 7 & N & N/A \\ \hline
    13 & 1 & N/A & N/A & 8 & Y & N/A \\ \hline
    14 & 1 & N/A & N/A & 8 & N & N/A \\ \hline
    15 & 1 & N/A & N/A & 9 & Y & N/A \\ \hline
    16 & 1 & N/A & N/A & 9 & N & N/A \\ \hline
    17 & 1 & N/A & N/A & 10 & Y & N/A \\ \hline
    18 & 1 & N/A & N/A & 10 & N & N/A \\ \hline
    19 & 1 & N/A & N/A & 11 & Y & N/A \\ \hline
    20 & 1 & N/A & N/A & 11 & N & N/A \\ \hline
    21 & 2 & N/A & N/A & 7 & Y & N/A \\ \hline
    22 & 2 & N/A & N/A & 7 & N & N/A \\ \hline
    23 & 2 & N/A & N/A & 8 & Y & N/A \\ \hline
    24 & 2 & N/A & N/A & 8 & N & N/A \\ \hline
    25 & 2 & N/A & N/A & 9 & Y & N/A \\ \hline
    26 & 2 & N/A & N/A & 9 & N & N/A \\ \hline
    27 & 2 & N/A & N/A & 10 & Y & N/A \\ \hline
    28 & 2 & N/A & N/A & 10 & N & N/A \\ \hline
    29 & 2 & N/A & N/A & 11 & Y & N/A \\ \hline
    30 & 2 & N/A & N/A & 11 & N & N/A \\ \hline
    31 & 3 & N/A & N/A & 7 & Y & N/A \\ \hline
    32 & 3 & N/A & N/A & 7 & N & N/A \\ \hline
    33 & 3 & N/A & N/A & 8 & Y & N/A \\ \hline
    34 & 3 & N/A & N/A & 8 & N & N/A \\ \hline
    35 & 3 & N/A & N/A & 9 & Y & N/A \\ \hline
    36 & 3 & N/A & N/A & 9 & N & N/A \\ \hline
    37 & 3 & N/A & N/A & 10 & Y & N/A \\ \hline
    38 & 3 & N/A & N/A & 10 & N & N/A \\ \hline
    39 & 3 & N/A & N/A & 11 & Y & N/A \\ \hline
    40 & 3 & N/A & N/A & 11 & N & N/A \\ \hline
    41 & 4 & N/A & N/A & 7 & Y & N/A \\ \hline
    42 & 4 & N/A & N/A & 7 & N & N/A \\ \hline
    43 & 4 & N/A & N/A & 8 & Y & N/A \\ \hline
    44 & 4 & N/A & N/A & 8 & N & N/A \\ \hline
    45 & 4 & N/A & N/A & 9 & Y & N/A \\ \hline
    46 & 4 & N/A & N/A & 9 & N & N/A \\ \hline
    47 & 4 & N/A & N/A & 10 & Y & N/A \\ \hline
    48 & 4 & N/A & N/A & 10 & N & N/A \\ \hline
    49 & 4 & N/A & N/A & 11 & Y & N/A \\ \hline
    50 & 4 & N/A & N/A & 11 & N & N/A \\ \hline
\end{tabularx}

\begin{tabularx}{\textwidth}{|X|X|X|X|X|X|X|}
    \caption{Test cases for output Y} \\ \hline
    \textbf{Test case} & \textbf{A} & \textbf{B} & \textbf{C} & \textbf{D} & \textbf{E} & \textbf{F} \\ \hline
    51 & N/A & A & N/A & N/A & N/A & N/A \\ \hline
    52 & N/A & B & N/A & N/A & N/A & N/A \\ \hline
    53 & N/A & C & N/A & N/A & N/A & N/A \\ \hline
    54 & N/A & D & N/A & N/A & N/A & N/A \\ \hline
    55 & N/A & E & N/A & N/A & N/A & N/A \\ \hline
\end{tabularx}

\begin{tabularx}{\textwidth}{|X|X|X|X|X|X|X|}
    \caption{Test cases for output Z} \\ \hline
    \textbf{Test case} & \textbf{A} & \textbf{B} & \textbf{C} & \textbf{D} & \textbf{E} & \textbf{F} \\ \hline
    56 & N/A & N/A & 100 & N/A & N/A & $\alpha$ \\ \hline
    57 & N/A & N/A & 100 & N/A & N/A & $\beta$ \\ \hline
    58 & N/A & N/A & 100 & N/A & N/A & $\gamma$ \\ \hline
    59 & N/A & N/A & 200 & N/A & N/A & $\alpha$ \\ \hline
    60 & N/A & N/A & 200 & N/A & N/A & $\beta$ \\ \hline
    61 & N/A & N/A & 200 & N/A & N/A & $\gamma$ \\ \hline
    62 & N/A & N/A & 300 & N/A & N/A & $\alpha$ \\ \hline
    63 & N/A & N/A & 300 & N/A & N/A & $\beta$ \\ \hline
    64 & N/A & N/A & 300 & N/A & N/A & $\gamma$ \\ \hline
\end{tabularx}
\end{centering}

\section*{Q3}
If we wanted to test all combinations using combinatorial testing, we would have 
\begin{align}
    \text{\#Test cases} &= 3 \times 3 \times 4 \times 3 \times 5 \times 4 \times 4 \times 5 \times 4 \\
    &= 172800.
\end{align}
Next, we wrote a Python script to automate generating the pairwise test cases. The result is shown below.

\scriptsize
\begin{tabularx}{\textwidth}{|X|X|X|X|X|X|X|X|X|X|}
    \caption{Results from Python using pairwise testing} \\ \hline
    \textbf{\#} & \textbf{HKBH} & \textbf{KBH} & \textbf{KB} & \textbf{NVH} & \textbf{NV} & \textbf{OR} & \textbf{SLL} & \textbf{SLS} & \textbf{TS} \\ \hline
    1 & NO & NO & 12KEY & NO & DPAD & LAND & MASK & LRG & FNGR \\ \hline
    2 & NDEF & NDEF & NOKEYS & NDEF & NONAV & PORT & NO & MASK & FNGR \\ \hline
    3 & YES & YES & QWERTY & YES & TRKBALL & SQR & NDEF & NORM & FNGR \\ \hline
    4 & YES & NDEF & NDEF & NO & NDEF & NDEF & YES & SML & NTOUCH \\ \hline
    5 & NDEF & NO & NDEF & YES & WHEEL & NDEF & NDEF & NDEF & STYLUS \\ \hline
    6 & NO & YES & NOKEYS & NDEF & WHEEL & SQR & YES & NDEF & NDEF \\ \hline
    7 & NO & NDEF & QWERTY & YES & NDEF & LAND & NO & LRG & NDEF \\ \hline
    8 & NDEF & YES & 12KEY & NO & NONAV & SQR & MASK & SML & STYLUS \\ \hline
    9 & YES & NO & NOKEYS & NDEF & TRKBALL & LAND & MASK & MASK & NTOUCH \\ \hline
    10 & NO & NO & QWERTY & NO & NONAV & PORT & YES & NORM & NTOUCH \\ \hline
    11 & YES & YES & 12KEY & YES & DPAD & PORT & NO & NDEF & NTOUCH \\ \hline
    12 & NDEF & NDEF & 12KEY & NDEF & DPAD & NDEF & NDEF & NORM & NDEF \\ \hline
    13 & NDEF & NDEF & QWERTY & NDEF & WHEEL & LAND & MASK & SML & STYLUS \\ \hline
    14 & NO & YES & NDEF & NDEF & NDEF & PORT & NDEF & LRG & STYLUS \\ \hline
    15 & NDEF & NO & NDEF & NO & TRKBALL & SQR & NO & SML & NDEF \\ \hline
    16 & YES & YES & NOKEYS & YES & NONAV & NDEF & NDEF & MASK & NDEF \\ \hline
    17 & NDEF & NDEF & NOKEYS & NO & NDEF & SQR & YES & LRG & NTOUCH \\ \hline
    18 & NO & NDEF & NDEF & NO & TRKBALL & NDEF & MASK & NDEF & FNGR \\ \hline
    19 & YES & YES & NDEF & NO & WHEEL & LAND & NDEF & NORM & NTOUCH \\ \hline
    20 & YES & NO & 12KEY & YES & NDEF & PORT & MASK & NORM & NDEF \\ \hline
    21 & YES & YES & 12KEY & YES & WHEEL & PORT & YES & MASK & STYLUS \\ \hline
    22 & NO & YES & NOKEYS & NO & DPAD & SQR & YES & MASK & STYLUS \\ \hline
    23 & YES & YES & QWERTY & YES & DPAD & NDEF & NO & SML & STYLUS \\ \hline
    24 & YES & YES & NDEF & NO & NONAV & LAND & YES & NDEF & FNGR \\ \hline
    25 & YES & YES & NDEF & NO & DPAD & NDEF & NDEF & LRG & STYLUS \\ \hline
    26 & NO & YES & NOKEYS & NO & TRKBALL & PORT & YES & SML & STYLUS \\ \hline
    27 & YES & YES & QWERTY & NO & WHEEL & PORT & NO & NORM & STYLUS \\ \hline
    28 & YES & YES & QWERTY & NO & NDEF & PORT & NDEF & NDEF & FNGR \\ \hline
    29 & YES & YES & NDEF & NO & WHEEL & PORT & NDEF & SML & FNGR \\ \hline
    30 & YES & YES & NDEF & NO & WHEEL & PORT & NDEF & MASK & STYLUS \\ \hline
    31 & YES & YES & QWERTY & NO & WHEEL & PORT & NDEF & MASK & STYLUS \\ \hline
    32 & YES & YES & NOKEYS & NO & WHEEL & PORT & NDEF & NORM & STYLUS \\ \hline
    33 & YES & YES & 12KEY & NO & TRKBALL & PORT & NDEF & LRG & STYLUS \\ \hline
    34 & YES & YES & NDEF & NO & WHEEL & PORT & NDEF & LRG & STYLUS \\ \hline
    35 & YES & YES & NDEF & NO & NDEF & PORT & NDEF & MASK & STYLUS \\ \hline
    36 & YES & YES & NDEF & NO & NONAV & PORT & NDEF & LRG & STYLUS \\ \hline
\end{tabularx}
\normalsize

As we can see from the above table, we have 36 test cases from pairwise testing. This is significantly less than the 172800 cases from combinatorial testing.

\section*{Q4}
The grammar can be tested with a single test case. An example of such a test case is
\begin{equation}
    abcdefghijklmnopqrstuvwxyz + A - 1 * B / CDEFGHIJKLMNOPQRSTUVWXYZ. 
\end{equation}

\subsection*{i)}
We have 4 different operators, 52 different letters, and 10 different digits. This means we have $4 + 52 + 10 = 66$ possible symbols. 

\subsection*{ii)}
\textlangle{}expr\textrangle{} has 3 permutations, \textlangle{}id\textrangle{} has 2 permutations, and \textlangle{}num\textrangle{} has 2 permutations. This means we have $3 + 2 + 2 + 66 = 73$ cases for production coverage.

\subsection*{iii)}
We can't perform derivation coverage because \textlangle{}id\textrangle{}::=\textlangle{}letter\textrangle{}$|$\textlangle{}letter\textrangle{}\textlangle{}id\textrangle{} leads to infinite tests.

\end{document}
